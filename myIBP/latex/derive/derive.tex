\documentclass{article}                                                   
\usepackage{fullpage}                                                     
\usepackage{pgffor}                                                       
\usepackage{amssymb}                                                      
%\usepackage{Sweave}                                                      
\usepackage{bm}                                                           
\usepackage{mathtools}                                                    
\usepackage{verbatim}                                                     
\usepackage{appendix}                                                     
\usepackage{graphicx}
\usepackage{float}
\usepackage{paralist} % compactenum
\usepackage[UKenglish]{isodate} % for: \today                             
\cleanlookdateon                % for: \today                             
                                                                          
\def\wl{\par \vspace{\baselineskip}\noindent}                             
%\def\beginmyfig{\begin{figure}[!Htbp]\begin{center}}                     
\def\beginmyfig{\begin{figure}[H]\begin{center}}                          
\def\endmyfig{\end{center}\end{figure}}                                   
\def\prodl#1#2#3{\prod\limits_{#1=#2}^{#3}}                               
\def\suml#1#2#3{\sum\limits_{#1=#2}^{#3}}                                 
\def\ds{\displaystyle}                                                    
\newcommand\numberthis{\addtocounter{equation}{1}\tag{\theequation}}
% Compactenum:
\renewenvironment{itemize}[1]{\begin{compactitem}#1}{\end{compactitem}}
\renewenvironment{enumerate}[1]{\begin{compactenum}#1}{\end{compactenum}}
\renewenvironment{description}[0]{\begin{compactdesc}}{\end{compactdesc}}
                                                                          
\begin{document}                                                          
% my title:                                                               
\begin{center}                                                            
  \section*{\textbf{Alternative Derivation For the pmf of the 
                    Indian Buffet Process?}}   %
  \subsection*{\textbf{Arthur Lui}}                                       
  \subsection*{\noindent\today}                                           
\end{center}                                                              
\setkeys{Gin}{width=0.5\textwidth}                                        
%%%%%%%%%%%%%%%%%%%%%%%%%%%%%%%%%%%%%%%%%%%%%%%%%%%%%%%%%%%%%%%%%%%%%%%%%%


\subsection{Definitions}
\noindent
Let:
\begin{itemize}
  \item $x_i$ be the number of \textit{new} dishes that customer $i$ draws.\\ 
  \item $y_{-i} = \suml{j}{1}{i-1} x_j$, the number of existing dishes
        before customer $i$ draws new dishes.
  \item $g_i(x_i) = \ds\frac{(\alpha/i)^{x_i}\exp(-\alpha/i)}{x_i!}$, 
        which is the probability mass function for a Poisson distribution 
        with parameter $\alpha/i$ and argument $x_i$.\\
  \item $h(i) = \prodl{k}{1}{y_{-i}} 
        \frac{(m_{-i,k})^{z_{i,k}}(i-m_{-i,k})^{1-z_{i,k}}}{i}$, for $i>1$,
        and $1$ for $i=1$.
  \item $K$ = the number of non-zero columns in $\bm Z$.\\
  \item $m_{-i,k}$ = the number of customers that took dish $k$ before customer
        customer $i$.
  \item $m_k$ = the \textit{total} number of customers that took dish $k$. 
        Note that for the last $x_N$ columns, $m_k$ is always $1$.
  \item $H_N = \suml{i}{1}{N} \ds\frac{1}{i}$.
\end{itemize}


\subsection{Description of IBP}
\begin{itemize}
  \item The first customer draws a $x_1=$ Pois($\alpha$) number of dishes. This
        is indicated by setting $z_{1,1:x_1}$ to $1$.
  \item For the next $i$ customers, for $i=1:N$, 
     \begin{itemize}
       \item if there exists previously sampled dishes, then customer $i$
             samples dish k  with probability $m_{-i,k}/i$, for $ k = 1:y_{-i}$.
     \end{itemize}
\end{itemize}


\subsection{Derivation}
\begin{align*}
  \text{P}(\bm Z)
  =&\prodl{i}{1}{N} \left\{g_i(x_i) \prodl{k}{1}{y_{-i}} 
    \frac{(m_{-i,k})^{z_{i,k}}(i-m_{-i,k})^{1-z_{i,k}}}{i}\right\}\\
  =&\prodl{i}{1}{N} \left\{ \ds\frac{(\alpha/i)^{x_i}\exp(-\alpha/i)}{x_i!} 
    \prodl{k}{1}{y_{-i}} 
    \ds\frac{(m_{-i,k})^{z_{i,k}}(i-m_{-i,k})^{1-z_{i,k}}}{i} \right\} \\
  =&\ds\frac{\alpha^{\suml{i}{1}{N}x_i} \exp(-\alpha H_N)} {\prodl{i}{1}{N}x_i!} 
    \left( \prodl{i}{1}{N}i^{-x_i} \right)
    \left( \prodl{i}{1}{N}\prodl{k}{1}{y_{-i}} 
    \ds\frac{(m_{-i,k})^{z_{i,k}}(i-m_{-i,k})^{1-z_{i,k}}}{i} \right)\\
  =&\ds\frac{\alpha^K \exp(-\alpha H_N)} {\prodl{i}{1}{N}x_i!} 
    \left( \prodl{i}{2}{N} i^{-x_i}\prodl{k}{1}{y_{-i}} 
    \ds\frac{(m_{-i,k})^{z_{i,k}}(i-m_{-i,k})^{1-z_{i,k}}}{i} \right)
    \numberthis\\
  \vdots&\\
  \vdots&\\
  =&\ds\frac{\alpha^K \exp(-\alpha H_N)} {\prodl{i}{1}{N}x_i!} 
    \left( \prodl{i}{1}{K} \ds\frac{(N-m_k)!(m_k-1)!}{N!} \right) \numberthis
\end{align*}
In simulation, (1) = (2).

\newpage
\subsection{Code for Evaluating pmfs:}
\begin{verbatim}
get.new.dish <- function(z) { # takes a Z matrix and returns the vector x
  N <- nrow(z)
  K <- ncol(z)
  x <- rep(0,N)
  x[1] <- sum(z[1,])
  for (i in 2:N) {
    if (sum(x[1:(i-1)])+1 <= K) x[i] <- sum(z[i,(sum(x[1:(i-1)])+1):K])
  }
  x
}

dibp <- function(Z,a) { # Original: Corresponds to eq.(2)
  N <- nrow(Z)
  K <- ncol(Z)
  Hn <- sum(1/1:N)
  x <- get.new.dish(Z)
  mk <- apply(Z,2,sum)
  
  p <- a^K / prod(gamma(x+1)) * exp(-a*Hn) *
       prod(gamma(N-mk+1)*gamma(mk)/gamma(N+1))
  p
}

dibp2 <- function(Z,a) { # Mine: Corresponds to eq.(1)
  N <- nrow(Z)
  K <- ncol(Z)
  Hn <- sum(1/1:N)
  x <- get.new.dish(Z)
  p <- a^K / prod(gamma(x+1)) * exp(-a*Hn)

  if (K>0) {
    for (i in 2:N) {
      p <- p / i^x[i]
      y <- sum(x[1:(i-1)])
      if (y>0) {
        for (k in 1:y) {# For each existing dish previous to customer i
          m_ik <- sum(Z[1:(i-1),k])
          p <- p * m_ik^Z[i,k] * (i-m_ik)^(1-Z[i,k]) / i
        }
      }
    }
  }

  p
}
\end{verbatim}

\newpage
\subsection{Code for Sampling from IBP}
\begin{verbatim}
rIBP <- function(N=3,a=1){
  K <- rpois(1,a) # The first customer draws POI(a) number of dishes
  Z <- matrix(0,N,K)
  Z[1,0:K] <- 1
  
  for (i in 2:N) {
    # Sample previously sampled dishes
    if (K>0) {
      # Each customer samples dish i with probability
      # mk / i, where mk is the current column sum
      mk <- apply(Z,2,sum) 
      pk <- mk / i
      Z[i,] <- pk > runif(K) 
    } 

    # Customer i samples an addition POI(a/i) new dishes
    newK <- K+rpois(1,a/i)
    col0 <- matrix(0,N,newK-K)
    if (ncol(col0) > 0) {
      Z <- cbind(Z,col0)
      Z[i,(K+1):newK] <- 1
      K <- newK
    }
  }

  Z
}
\end{verbatim}

Example simulations are in the "hongkong/myIBP" directory.
\end{document}

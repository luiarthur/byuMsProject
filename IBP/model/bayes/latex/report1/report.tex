%http://www.informatik.uni-freiburg.de/~frank/ENG/latex-course/latex-course-3/latex-course-3_en.html
\documentclass{beamer}

\usepackage{graphicx}
\usepackage{textpos}
\usepackage{amsmath}
\usepackage{bm}
\usepackage{color} % For my tc command
\usepackage[labelformat=empty]{caption}
\def\wl{\par \vspace{\baselineskip}}
\newcommand\tc[1]{\textcolor{red}{\textbf{#1}}}

% See this for more themes and colors: http://www.hartwork.org/beamer-theme-matrix/
\usepackage{beamerthemeHannover} % Determines the Theme
\usecolortheme{seahorse}         % Determines the Color

\title{Inference Using The Indian Buffet Process}
\logo{\includegraphics[width=1cm,height=1cm,keepspectration]{/data/arthurll/Pictures/logo.png}}

\author[Arthur Lui]{Arthur L. Lui}
\institute[Brigham Young University]{
  Department of Statistics\\
  Brigham Young University
}
%Setting %%%%%%%%%%%%%%%%%%%%%%%%%%%%%%%%%%%%%%%%%%%%%%%%%%%%%%%%%%%%%%%%%%%%%%%%%



\begin{document}
  
  \frame{\titlepage}

  \section{Introduction}
  \begin{frame}
  \frametitle{The Indian Buffet Process}
    \begin{itemize}
      %\item IBP has many applications
      \item Extension of the Chinese restaurant process
      \item Defines a distribution for sparse binary matrices
      \item IBP effectively recovers latent structure responsible for generating 
            observed data
    \end{itemize}
  \end{frame}

  \section{Motivating Example}
  \begin{frame}
  \frametitle{Motivating Example}
     Suppose you observe a matrix $\bm X_{n \times d}$ = \wl

       \begin{tabular}{l|c|c|c|c|c|}
         \multicolumn{1}{c}{}
           & \multicolumn{1}{c}{Arthur} 
           & \multicolumn{1}{c}{Mickey} 
           & \multicolumn{1}{c}{Minnie}
           & \multicolumn{1}{c}{Phil} 
           & \multicolumn{1}{c}{Ryan} \\ 
         \cline{2-6}

         Hunger Games & 18 & 24 & 28 & 20 & 25 \\ \cline{2-6}
         Ender's Game & 29 & 27 & 26 & 24 & 29 \\ \cline{2-6}
         Hobbit       & 20 & 16 & 16 & 12 & 14 \\ \cline{2-6}
       \end{tabular}

       \wl\noindent
       where $n=3$ and $d=5$.
       \begin{itemize}
         \item $\bm X = \bm Z \bm A$ \\
         \item $\bm Z_{n \times k}$ is some binary feature matrix and 
               $\bm A_{k \times d}$ is some weight matrix.
       \end{itemize}
  \end{frame}
  
  \begin{frame}
  \frametitle{Motivating Example}
      Suppose that the latent variables $\bm Z$ and $\bm A$ are:\\ 
        \wl
        $\bm Z =$
        \begin{tabular}{l|c|c|c|c|c|c|c|c|}
         \multicolumn{1}{c}{}
           % These may represent time periods of music: 
           % Baroque, Classical, Romantic, 20th Century, etc.
           & \multicolumn{1}{c}{A} 
           & \multicolumn{1}{c}{B}
           & \multicolumn{1}{c}{C}
           & \multicolumn{1}{c}{D}
           & \multicolumn{1}{c}{E}
           & \multicolumn{1}{c}{F}
           & \multicolumn{1}{c}{G}
           & \multicolumn{1}{c}{H} \\ 
         \cline{2-9}

         Hunger Games &1&1&1&1&1&0&0&0 \\ \cline{2-9}
         Ender's Game &1&1&1&1&0&1&1&0 \\ \cline{2-9}
         Hobbit       &0&1&0&1&0&0&0&1 \\ \cline{2-9}
       \end{tabular}
       \wl
       $\bm A =$
       \begin{tabular}{l|c|c|c|c|c|}
         \multicolumn{1}{c}{}
           & \multicolumn{1}{c}{Arthur} 
           & \multicolumn{1}{c}{Mickey}
           & \multicolumn{1}{c}{Minnie}
           & \multicolumn{1}{c}{Phil}
           & \multicolumn{1}{c}{Ryan} \\ 
         \cline{2-6}

         A & 6&2&6&4&3\\ \cline{2-6}
         B & 6&7&9&1&4\\ \cline{2-6}
         C & 1&4&4&3&7\\ \cline{2-6}
         D & 3&5&2&6&6\\ \cline{2-6}
         E & 2&6&7&6&5\\ \cline{2-6}
         F & 9&4&2&5&5\\ \cline{2-6}
         G & 4&5&3&5&4\\ \cline{2-6}
         H &11&4&5&5&4\\ \cline{2-6}
       \end{tabular}
       
       % Z_{ij} indicates whether composer i is in time period / genre j.
       % A_{jk} is some weight matrix telling us how likely (out of 30) 
       % is it for person k to time period / genre j.
  \end{frame}

  \begin{frame}
  % This frame is a repeat of the previous with subs for ABC..H
  \frametitle{Motivating Example}
      Suppose:
        \wl
        $\bm Z =$
        \begin{table}
        \resizebox{\columnwidth}{!}{%
          \begin{tabular}{l|c|c|c|c|c|c|c|c|}
            \multicolumn{1}{c}{}
              % These may represent time periods of music: 
              % Baroque, Classical, Romantic, 20th Century, etc.
              & \multicolumn{1}{c}{Kids} 
              & \multicolumn{1}{c}{Action}
              & \multicolumn{1}{c}{Future}
              & \multicolumn{1}{c}{YA}
              & \multicolumn{1}{c}{Female}
              & \multicolumn{1}{c}{Space}
              & \multicolumn{1}{c}{Aliens}
              & \multicolumn{1}{c}{Wizards} \\ 
            \cline{2-9}

            Hunger Games &1&1&1&1&1&0&0&0 \\ \cline{2-9}
            Ender's Game &1&1&1&1&0&1&1&0 \\ \cline{2-9}
            Hobbit       &0&1&0&1&0&0&0&1 \\ \cline{2-9}
         \end{tabular}%
       }
       \end{table}
       \wl
       $\bm A =$
       \begin{table}
       \tiny{%  
         \begin{tabular}{l|c|c|c|c|c|}
           \multicolumn{1}{c}{}
             & \multicolumn{1}{c}{Arthur} 
             & \multicolumn{1}{c}{Mickey}
             & \multicolumn{1}{c}{Minnie}
             & \multicolumn{1}{c}{Phil}
             & \multicolumn{1}{c}{Ryan} \\ 
           \cline{2-6}

           Kids   & 6&2&6&4&3\\ \cline{2-6}
           Action & 6&7&9&1&4\\ \cline{2-6}
           Future & 1&4&4&3&7\\ \cline{2-6}
           YA     & 3&5&2&6&6\\ \cline{2-6}
           Female & 2&6&7&6&5\\ \cline{2-6}
           Space  & 9&4&2&5&5\\ \cline{2-6}
           Aliens & 4&5&3&5&4\\ \cline{2-6}
           Wizard &11&4&5&5&4\\ \cline{2-6}
         \end{tabular}%
       }  
      \end{table} 
       % This section is a repeat of the previous
  \end{frame}



  \section{The Indian Buffet Process}
  \begin{frame}
  \frametitle{Left Ordered Form}
    $lof(\cdot)$ is a many-to-one function.

    \begin{figure}
      \centering
      \begin{minipage}{.5\textwidth}
        \centering
        \includegraphics[width=.4\linewidth]{pics/Z.pdf}
        \caption{M}
      \end{minipage}% The percent keeps the two graphs on same line
      \begin{minipage}{.5\textwidth}
        \centering
        \includegraphics[width=.4\linewidth]{pics/lofZ.pdf}
        \caption{lof(M)}
      \end{minipage}
    \end{figure}  

  \end{frame}

  \begin{frame}
  \frametitle{Indian Buffet Process}
    \begin{center}\begin{figure}
      \includegraphics[width=7cm]{pics/draw1.pdf}
    \end{figure}\end{center}  
  \end{frame}

  \begin{frame}
  \frametitle{Indian Buffet Process}
    \begin{center}\begin{figure}
      \includegraphics[width=7cm]{pics/draw2.pdf}
    \end{figure}\end{center}  
  \end{frame}

  \begin{frame}
  \frametitle{Indian Buffet Process}
    \begin{center}\begin{figure}
      \includegraphics[width=7cm]{pics/exp.pdf}
    \end{figure}\end{center}  
  \end{frame}

  \begin{frame}
  \frametitle{Indian Buffet Process Prior}
    \[
      P([\bm Z]) = \frac{\alpha^{K_+}}{\prod_{h=1}^{2^{N-1}}K_h!} e^{-\alpha H_N} 
                   \prod_{k=1}^{K_+} \frac{(N-m_k)!(m_k-1)!}{N!}
    \]
  \end{frame}

  \begin{frame}
  \frametitle{Model}
    \[ \underset{n \times d}{\bm X} | \bm{Z}, \bm{A} \sim \text{MN}(\bm{ZA,U,V})\]
    \[ \underset{n \times k}{\bm Z} \sim \text{IBP}(\alpha) \]
    \[ \underset{k \times d}{\bm A} \sim \text{MN}(\bm{M,I_k,I_d})\]
    where $\bm X \sim \text{MN}(\bm{M,U,V}) \iff 
          \text{vec}(\bm{X}) \sim 
          \text{N}_{np}(\text{vec}(\bm M),\bm V \otimes \bm U)$
  \end{frame}


  \section{Results}
  \begin{frame}
  \frametitle{Results}
  Posterior Mode for $\bm Z$ = 
        \begin{tabular}{l|c|c|c|c|c|c|c|c|}
         \multicolumn{1}{c}{}
           % These may represent time periods of music: 
           % Baroque, Classical, Romantic, 20th Century, etc.
           & \multicolumn{1}{c}{A} 
           & \multicolumn{1}{c}{B}
           & \multicolumn{1}{c}{C}
           & \multicolumn{1}{c}{D}
           & \multicolumn{1}{c}{E}
           & \multicolumn{1}{c}{F}
           & \multicolumn{1}{c}{G}
           & \multicolumn{1}{c}{H} \\ 
         \cline{2-9}

        Hunger Games &1&1&   1&1&   1&\tc1&   0&0 \\ \cline{2-9}
        Ender's Game &1&1&   1&1&\tc1&   1&   1&0 \\ \cline{2-9}
        Hobbit       &0&1&\tc1&1&   0&   0&\tc1&1 \\ \cline{2-9}

       \end{tabular}

  \wl
  Posterior Mean for $\bm A$ = \\ \wl
       \begin{tabular}{l|r|r|r|r|r|}
         \multicolumn{1}{c}{}
           & \multicolumn{1}{c}{Arthur} 
           & \multicolumn{1}{c}{Mickey}
           & \multicolumn{1}{c}{Minnie}
           & \multicolumn{1}{c}{Phil}
           & \multicolumn{1}{c}{Ryan} \\ 
         \cline{2-6}

         A &3  (6)&5 (2)&5 (6)&3  (4)&5  (3)\\ \cline{2-6}
         B &4  (6)&5 (7)&6 (9)&5  (1)&6  (4)\\ \cline{2-6}
         C &3  (1)&2 (4)&4 (4)&0  (3)&4  (7)\\ \cline{2-6}
         D &3  (3)&4 (5)&4 (2)&4  (6)&3  (6)\\ \cline{2-6}
         E &3  (2)&3 (6)&4 (7)&4  (6)&5  (5)\\ \cline{2-6}
         F &3  (9)&2 (4)&3 (2)&2  (5)&1  (5)\\ \cline{2-6}
         G &9  (4)&4 (5)&2 (3)&4  (5)&4  (4)\\ \cline{2-6}
         H &1 (11)&3 (4)&0 (5)&-1 (5)&-1 (4)\\ \cline{2-6}
       \end{tabular}

  \end{frame}
  

  \begin{frame}
  \frametitle{Results}
    The product of the two matrices $\bm{\hat{Z}}\bm{\hat{A}}=\bm{\hat X}=$ \\
       \begin{tabular}{l|c|c|c|c|c|}
         \multicolumn{1}{c}{}
           & \multicolumn{1}{c}{Arthur} 
           & \multicolumn{1}{c}{Mickey} 
           & \multicolumn{1}{c}{Minnie}
           & \multicolumn{1}{c}{Phil} 
           & \multicolumn{1}{c}{Ryan} \\ 
         \cline{2-6}

         Hunger Games &17.78&21.38&26.63&19.00&24.06 \\ \cline{2-6}
         Ender's Game &26.32&25.65&28.99&23.33&27.60 \\ \cline{2-6}
         Hobbit       &18.90&17.59&16.81&12.87&15.41 \\ \cline{2-6}
       \end{tabular}
     \wl
     The original $\bm X$ = \\
       \begin{tabular}{l|c|c|c|c|c|}
         \multicolumn{1}{c}{}
           & \multicolumn{1}{c}{Arthur} 
           & \multicolumn{1}{c}{Mickey} 
           & \multicolumn{1}{c}{Minnie}
           & \multicolumn{1}{c}{Phil} 
           & \multicolumn{1}{c}{Ryan} \\ 
         \cline{2-6}

         Hunger Games & 18 & 24 & 28 & 20 & 25 \\ \cline{2-6}
         Ender's Game & 29 & 27 & 26 & 24 & 29 \\ \cline{2-6}
         Hobbit       & 20 & 16 & 16 & 12 & 14 \\ \cline{2-6}
       \end{tabular}
  \end{frame}


  \section{Future}
  \begin{frame}
  \frametitle{Future}
    \begin{itemize}
      \item $\bm Z$ was recovered relatively accurately
      \item $\bm A$ was not
      \item $\bm{\hat X} \approx \bm X$ 
      %\item Cross Validate to obtain features
      \item Distance-dependent IBP
      \item Prediction
    \end{itemize}

  \end{frame}

\end{document}    

\documentclass{article}
\usepackage{fullpage}
\usepackage{amssymb}
\usepackage{Sweave}
\usepackage{bm}
\usepackage{mathtools}
\def\wl{\par \vspace{\baselineskip}}
\def\imply{\Rightarrow}

\begin{document}

\begin{center}
  \section*{Inference Using The Indian Buffet Process}
  \subsection*{Arthur Lui}
  \subsection*{Department of Statistics}
\end{center}
\wl

\subsection*{Abstract}
  % What is the IBP, and what are its applications?

  % Bayesian non-parametric 
  % Applications in Clustering 
  % Machine Learning -> Statistical Learning

  The Indian buffet process (IBP) has recently been applied in various areas,
  including biochemistry, economics, and sociology. The IBP, an extension of the
  Chinese restaurant process, defines a prior distribution for latent feature
  models, a class of models in which observations are the result of multiple
  binary features. In this presentation, I will review the IBP and demonstrate
  its effectiveness in recovering the latent (i.e.\ hidden) structure
  responsible for generating observed data. Inference through Markov chain Monte
  Carlo (MCMC) will be presented for a matrix normal latent feature model. I
  will show the IBP's ability to unmix hidden sources from the observed data to
  reveal latent structure, even when the number of hidden sources is unknown.

\end{document}


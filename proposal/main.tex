%\documentclass[manuscript]{biometrika}
%\documentclass[lineno]{biometrika}
\documentclass[lineno]{biometrika-dbd}

\usepackage{amsmath}

%% Please use the following statements for
%% managing the text and math fonts for your papers:
\usepackage{times}
%\usepackage[cmbold]{mathtime}
\usepackage{bm}
\usepackage{natbib}
\usepackage{subcaption}

\usepackage[plain,noend]{algorithm2e}

\makeatletter
\renewcommand{\algocf@captiontext}[2]{#1\algocf@typo. \AlCapFnt{}#2} % text of caption
\renewcommand{\AlTitleFnt}[1]{#1\unskip}% default definition
\def\@algocf@capt@plain{top}
\renewcommand{\algocf@makecaption}[2]{%
  \addtolength{\hsize}{\algomargin}%
  \sbox\@tempboxa{\algocf@captiontext{#1}{#2}}%
  \ifdim\wd\@tempboxa >\hsize%     % if caption is longer than a line
    \hskip .5\algomargin%
    \parbox[t]{\hsize}{\algocf@captiontext{#1}{#2}}% then caption is not centered
  \else%
    \global\@minipagefalse%
    \hbox to\hsize{\box\@tempboxa}% else caption is centered
  \fi%
  \addtolength{\hsize}{-\algomargin}%
}
\makeatother

%%% User-defined macros should be placed here, but keep them to a minimum.
\usepackage{etoolbox}
    \newtoggle{draft}
    \toggletrue{draft}
    \togglefalse{draft}
	\newcommand{\draft}[1]{\iftoggle{draft}{#1}{}}
\usepackage{url}
\usepackage{setspace}
\usepackage{graphicx}
\usepackage{multirow}
\usepackage{booktabs}
%/\usepackage[thinlines]{easytable}
%\usepackage{ifthen}
%\usepackage[show]{r2r}
%\usepackage{soul}

\newcommand{\partition}{\ensuremath{\pi}}
\newcommand{\mass}{\alpha}
\newcommand{\given}{\mid}
\newcommand{\sizeof}[1]{| #1 |}
\newcommand{\I}[1]{\ensuremath{\text{\textnormal{I}}\{#1\}}}
\newcommand{\redact}[1]{}
\newcommand{\todo}[1]{{\color{red} #1}}
\newcommand{\DBD}[1]{\draft{{\color{red} DBD: #1}}}

% Not needed in production
%\draft{
\usepackage{color}
\usepackage[absolute]{textpos}
\setlength{\TPHorizModule}{1in}
\setlength{\TPVertModule}{\TPHorizModule}
\textblockorigin{0in}{0in} % start everything at the top-left corner
%}

\AtBeginDocument{
  \DeclareGraphicsExtensions{.png,.jpg,.pdf}
  \graphicspath{{../figures/}}
}

\begin{document}

\jname{Biometrika}
%% The year, volume, and number are determined on publication
\jyear{2013}
\jvol{100}
\jnum{1}
%% The \doi{...} and \accessdate commands are used by the production team
%\doi{10.1093/biomet/asm023}
\accessdate{Advance Access publication on DAY MONTH YEAR}
\copyrightinfo{\Copyright\ 2013 Biometrika Trust\goodbreak {\em Printed in Great Britain}}

%% These dates are usually set by the production team
\received{March 2013}
\revised{MONTH YEAR}

%% The left and right page headers are defined here:
\markboth{David B.\ Dahl, Hyun Joo, \and Jerry W.\ Tsai}{Partition Distributions Indexed by Pairwise Information}

%% Here are the title, author names and addresses
\title{Partition Distributions Indexed by Pairwise Information}

\author{DAVID B.\ DAHL}
\affil{Department of Statistics, Brigham Young University, Provo, Utah 84602, U.S.A. \email{dahl@stat.byu.edu}}

\author{HYUN JOO}
\affil{Department\ of Chemistry, University of the Pacific, Stockton, California 95211, U.S.A. \email{hjoo@pacific.edu}}

\author{JERRY W.\ TSAI}
\affil{Department\ of Chemistry, University of the Pacific, Stockton, California 95211, U.S.A. \email{jtsai@pacific.edu}}

\maketitle
%{\renewcommand{\baselinestretch}{1} \maketitle }

%\draft{
\begin{textblock}{8}(6.5,1.2)
\todo{Revision: 2013-06-10}
\end{textblock}
%}

\begin{abstract}
We propose a random partition distribution indexed by pairwise information such
that items which are ``close'' to each other are more likely to be clustered
together.  In contrast, the Dirichlet process induces a partition distribution
in which the probability that an item is clustered with another is uniform
across all items.  Our proposed distribution is a natural extension of the
Chinese restaurant process.  We show that standard Bayesian nonparametric
models are readily modified to use our random partition distribution based on
pairwise information.  The properties of our distribution are explored.
Applying our distribution as prior clustering distributions in a model for
protein structure prediction, we find that the methods incorporating pairwise
information substantially improve predictive accuracy.
\end{abstract}

\begin{keywords}
Bayesian nonparametrics;
Clustering method;
Dirichlet process mixture model;
Non-exchangeable prior;
Protein structure prediction;
Random partition model.
\end{keywords}

\section{Introduction}

Let $p(\partition_n)$ denote a probability distribution for a random partition
$\partition_n = \{S_1,\ldots,S_{q_n}\}$ of $\{1,\ldots,n\}$.  Popular models for
$p(\partition_n)$ include product partition models
\citep[see][]{hart:1990,barr:hart:1992}, species sampling models
\citep[see][]{pitm:1995,pitm:yor:1997}, and model-based clustering
\citep{rich:gree:1997,fral:raft:2002}.  Others include those of
\citet{pitm:2003} and \citet{lijo:mena:prns:2005}.  These models are
exchangeable with respect to permutations of the indices of the items, and many
of them are reviewed by \citet{quin:2006} and \citet{lau:gree:2007}. These
random probability models are routinely used in nonparametric Bayesian
analysis.

Exchangeable probability models for partitions have many attractive properties
and mathematical elegance, yet there are severe constraints on such models
\citep{leej:muel:trip:quin:2010}.  The aim of this paper is to explore random
probability models for partitions that utilize available pairwise information
to influence the clustering of the items.  Our models are, therefore,
explicitly non-exchangeable.  The use of pairwise distances is common in many
\emph{ad hoc} clustering algorithms (e.g., $k$-means and hierarchical
clustering).  Our method provides a means to incorporate this type of
information in nonparametric models and model-based clustering procedures.  We
show how to utilize this new distance-based random partition distribution as a
prior clustering distribution in Bayesian nonparametric models, thus achieving
a nonparametric framework that has both distance-based and model-based
clustering features.

Simultaneous work has developed other nonexchangeable probability models for
Bayesian nonparametric methods
\citep[e.g.,][]{park:duns:2007,shah:neal:2007,muel:quin:rosn:2008}. A common
thread is the use of a random probability for partitions influenced \textit{a
priori} by covariates.  These methods are reviewed by \citet{muel:quin:2008}.
\citet{park:duns:2007} and \citet{shah:neal:2007} include clustering covariates
as part of an augmented response vector to obtain a prior partition model for
inference on the response data.  \citet{park:duns:2007} build on product
partition models and focus on continuous covariates treated as random
variables, whereas \citet{shah:neal:2007} use the Dirichlet process as the
random partition and model a categorical response with logistic regression.
\citet{muel:quin:2008} also build on product partition models and can
accommodate both continuous and discrete covariates. In their simulation study
of several of these approaches, \citet{muel:quin:2008} found no dominant method
and suggest choosing among them based on the inferential goals.  Our approach
in this paper is unique, however, being tailored for pairwise distances,
providing two methods to incorporate distance-based similarity information in
both nonparametric models and model-based clustering procedures.  Our framework
is general in that any distance metric (defined on continuous, discrete, or
mixed covariates) can be used.  We demonstrate our proposal in the context of
protein structure prediction and show that our methodology substantially
improves predictive accuracy.

%  \item Blei, Frazier (2011): \textit{Distance dependent Chinese restaurant processes}
%  \item M\"uller, Quintana, Rosner (2011), Le\'on Novelo, et.\ al.\ (2011): Product partition model with regression on covariates
%  \item Guindani, et. al.\ (2011): Generalized Ottawa Sequences
%  \item Rodriguez, et.\ al. (2010), Chung, Dunson (2009): Probit stick breaking
%  \item Park, Dunson (2010): \textit{Bayesian Generalized Product Partition Model}
%  \item Dunson, Pillai, Park (2007): \textit{Bayesian density regression} (Weighted mixture of DP)
%  \item ``Mixture of Experts'': {\scriptsize Models with covariate-dependent weights}


\redact{
The paper is organized as follows.  The probability distribution on
partitions that is induced by the Dirichlet process is described in Section
\ref{sec_dp}.  In Section \ref{sec_attraction} we show how to modify the usual
Dirichlet process algorithms to yield the Attraction distribution for a
distance-based probability distribution on partitions.  The Affinity
distribution --- an alternative distance-based probability distribution
on partitions --- is described in Section \ref{sec_affinity}.  In Section
\ref{sec_properties}, we explore the properties of the proposed distributions
and show that a new class of Bayesian nonparametric models is available when
using these distributions as a prior clustering distribution.  The method is
illustrated in Section \ref{sec_mobility} in an example of protein structure
prediction and is shown to be substantially better than the leading alternative.
We conclude in Section \ref{sec_conclusion} by discussing computational
efficiency, other applications, and extensions of the proposed methodology.
}

\section{Chinese Restaurant Process}
\label{sec_cr}

\subsection{Notation}
\label{sec_notation}

Our proposed partition distribution indexed by pairwise information is an
adaptation of the partition distribution of the Chinese restaurant process.  We
lay the groundwork for our distribution by introducing some notation, defining
the Chinese restaurant process, and exploring some of its properties.

A partition $\partition_n = \{S_1,\ldots,S_{q_n}\}$ of $\{1,\ldots,n\}$ has the
following properties: (i) $S_i \not= \emptyset$ for $i=1,\ldots,q_n$ (i.e.,
non-empty subsets), (ii) $S_i \cap S_j = \emptyset$ for $i \not= j$ (i.e.,
mutually exclusive subsets), and (iii) $\cup_{j=1}^q S_j = \{1,\ldots,n\}$
(i.e., exhaustive subsets).  The number of subsets $q_n$ for a partition
$\partition_n$ can range from 1 (i.e., all items belong to the same subset) to
$n$ (i.e., each item is in a singleton subset).  Let $\mathcal{F}_n$ denote all
possible partitions of $\{1,\ldots,n\}$.  It is sometimes convenient to
represent a partition $\partition_n$ as a vector $\bm{c} = (c_1,\ldots,c_n)$ of
cluster labels, where $c_i = j$ if and only if $i \in S_j$, for $i=1,\ldots,n$
and $j=1,\ldots,q_n$.  The size of $\mathcal{F}_n$ grows exponentially in $n$
according to the \citet{bell:1934} number.  A probability distribution over
$\mathcal{F}_n$ is discrete, but the size of the space makes exhaustive
calculations impossible except for very small $n$.

\subsection{Sequentially Seating Customers using the Chinese Restaurant Process}
\label{sec_cr_sequential}

The Chinese restaurant process is a discrete-time stochastic process on the
positive integers described metaphorically as sequentially seating customers of
a restaurant at one of an infinite number of tables, each of infinite capacity.
For generality and the later exposition of our proposed distribution, we
describe the Chinese restaurant process where the order in which customers are
seated is not necessarily their order in the dataset.  That is, there is a
permutation $\bm{\sigma} = (\sigma_1,\ldots,\sigma_n)$ of the integers
$\{1,\ldots,n\}$ giving the sequence in which the $n$ customers are seated at
the restaurant, where the $i^{th}$ customer seated is $\sigma_i$ who has
cluster label $c_{\sigma_i}$.  The inverse is $\bm{\sigma}^{-1} =
(\sigma_1^{-1},\ldots,\sigma_n^{-1})$, where $\sigma_i^{-1}$ gives the order in
which customer $i$ is seated.  For example, if $\bm{\sigma} = (4,2,1,3)$, then
$\bm{\sigma}^{-1} = (3,2,4,1)$.

At time $i=1$, the first customer $\sigma_1$ in the permutation $\bm{\sigma}$
is seated at a table.  At time $i$, customer $\sigma_i$ is seated at an empty
table with probability $\mass / (\mass + i - 1)$ or at the table of an existing
customer with probability $(i-1) / (\mass + i -1 )$.  Further, the probability
of seating with each existing customer is uniform. After seating all $n$
customers, there are $q_n$ occupied tables and each table represents a subset
$S$ in the random partition $\partition_n = \{S_1,\ldots,S_q\}$.  In the
context of clustering, we will refer to customers as \emph{items}, tables as
\emph{clusters}, and the seating of the entire restuarant as the
\emph{clustering}.

The probability mass
function of the joint distribution of a cluster label vector $\bm{c}$, given
the permutation $\bm{\sigma}$, is the product of increasing conditionals:
\begin{equation}
\label{eq_dpm_ppf}
p(\bm{c} \given \bm{\sigma}) = \prod_{i=1}^n p(c_{\sigma_i} \given c_{\sigma_1}, \ldots, c_{\sigma_{i-1}})
\end{equation}
where:
\begin{equation}
\label{eq_dpm_ppf_component}
\begin{split}
p(c_{\sigma_i} \given c_{\sigma_1},\ldots,c_{\sigma_{i-1}}) 
  & \ = \ \text{pr}(c_{\sigma_i} = j \given c_{\sigma_1},\ldots,c_{\sigma_{i-1}}) \\
  & \ \propto \
\begin{cases}
\ \sum_{k=1}^{i-1} \ \text{I}\{c_{\sigma_k} = j\} & \text{for\
\ } j=1,\ldots,q_{i-1} \\
\ \ \mass & \text{for\ \ } j = q_{i-1} + 1,
\end{cases}
\end{split}
\end{equation}
where the normalizing constant is $1/( \mass + i - 1 )$ and $q_{i-1}$ is the
number of occupied clusters after placing the first $i-1$ items, and
$\text{I}\{ \cdot \}$ is the indicator function, so $\sum_{k=1}^{i-1}
\text{I}\{c_{\sigma_k} = j\}$ is the number of items already placed at
cluster $j$ before time $i$.  Note that the probability that item $\sigma_i$ is
assigned to a new cluster is $\mass / ( \mass + i - 1)$.

The resulting partition distribution is often called the Ewens-Pitman
distribution \citep{ewens:1972,pitm:1995,pitm:1996}.  The metaphor to a Chinese
restaurant first appears in \citet[page 91-92]{aldo:1985} and is credited to
Jim Pitman and Lester E.\ Dubins.  This distribution is the allocation
distribution of the Dirichlet process \citep{ferg:1973}, which is commonly used
in Bayesian nonparametric models.

\subsection{Properties of the Chinese Restaurant Process}
\label{sec_cr_exchangeability}

Note that the allocation probabilities of new items to existing clusters in the
Chinese restaurant process depend only on the number of items already assigned
at each cluster.  Therefore, the Chinese restaurant process is exchangeable
\citep{pitm:1995}; for any cluster label vector $\bm{c}$ and permutation
$\bm{\sigma}$, the probability of an arbitrary permutation $\bm{c}^\ast$ of the
elements of $\bm{\sigma}$ is equal to the probability of $\bm{\sigma}$.
Further, the probability of any cluster label vector $\bm{c}$ --- or,
equivalently, any partition $\partition_n$ --- is unaffected by the order
$\bm{\sigma}$ in which the items are assigned.  Formally, for two permutations
$\bm{\sigma}^{(1)}$ and $\bm{\sigma}^{(2)}$ among all possible $n!$
permutations and any partition $\partition_n$, we have:
\begin{equation}
\label{eq_permutation_invariance}
p(\partition_n \given \bm{\sigma}^{(1)}) = p(\partition_n \given \bm{\sigma}^{(2)}).
\end{equation}
We can thus drop the conditioning on the permutation $\bm{\sigma}$ and write
simply $p(\bm{c})$ or $p(\partition_n)$.  Exchangeability establishes that the
marginal probability that any two items $i$ and $k$ are clustered together is:
\begin{equation}
\label{eq_uniform}
\text{pr}(c_i = c_k ) = \frac{1}{\mass+1},
\end{equation}
and this probability is uniform across all pairs $i$ and $k$.  The probability
mass function of a partition $\partition_n$ for the Chinese restaurant process
is:
\begin{equation}
\label{eq_dpm}
p(\partition_n) = \prod_{S \in \partition_n} \mass \Gamma(\sizeof{S}) \Biggg/ \mass^{(n)},
\end{equation}
where $\mass^{(n)} = \prod_{i=1}^n(\mass+i-1)$ is the reciprocal of the
normalizing constant, $\Gamma(x)$ is the gamma function, and $\sizeof{S}$ is
the number of items in subset $S$.

The Chinese restaurant process is also consistent in the following sense.
Consider two distributions of the partition $\partition_{n}$ of the $n$ items:
1.\ The distribution $p_1$ described by (\ref{eq_dpm}), and 2.\ The
distribution $p_2$ obtained by marginalizing the distribution of the partition
$\partition_{n+1}$ of $n+1$ items in (\ref{eq_dpm}) over item $n+1$.  For the
Chinese restaurant process, these two distributions $p_1$ and $p_2$ are the
same and we say that the Chinese restaurant process is also consistent.

The mass parameter $\mass$ governs the probability of opening a new cluster
and, therefore, the distribution of the total number $q_n$ of occupied clusters
after assigning $n$ items.  Specifically:
\begin{equation}
\label{eq_dist_number_of_clusters}
\text{pr}(q_n = k) = \frac{ \mass^k \sizeof{s(n,k)} }{ \mass^{(n)} } \hspace{3ex} \text{for} \ k=1,\ldots,n,
\end{equation}
where $s(n,k)$ is the Stirling number of the first kind counting the number of
arrangments of $n$ objects into $k$ non-empty circular permutations.  Further,
\citet{arra2003} shows that:
\begin{equation}
\label{eq_moments_number_of_clusters}
\text{E}(q_n) = \sum_{i=1}^n \frac{\mass}{\mass+i-1} \hspace{3ex} \text{and} \hspace{3ex}
\text{var}(q_n) = \mass \sum_{i=1}^n \frac{i-1}{(\mass+i-1)^2}.
\end{equation}
The Chinese restaurant process is consistent across dimensions in the sense
that the partition distribution of the first $n-1$ items obtained by removing
item $n$ from the random partition is the same as the partition distribution at
time $n-1$ \citep[page 92]{aldo:1985}.

\section{Nonexchangeable Partition Distributions Indexed by Pairwise Information}
\label{sec_proposal}

\subsection{Pairwise Information as Distances or Proximities}
\label{sec_proposal_proximities}

We proposed a random partition distribution indexed by pairwise information.
We suppose that the prior information regarding clustering of $n$ items can be
expressed in terms of an $n \times n$ proximity matrix of elements $s_{ik}$
giving the proximity between items $i$ and $k$.  If pairwise information
regarding $i$ and $k$ is more readily expressed in terms of a distance
$d_{ik}$, then the proximity $s_{ik}$ can be expressed as a function of
$d_{ik}$, e.g., $s_{ik} = 1/d_{ik}$ or $s_{ik} = \max_{\forall ab}\{d_{ab}\} -
d_{ik}$.  Our framework can use any proximity or distance metric such that $0 <
s_{ik} < \infty$ for all $i,k$.

Recall that exchangeability imposes uniform probabilities in (\ref{eq_uniform})
on all pairs of items.  In the presence of proximity or distance information,
exchangeability is too limiting.  We desire a nonexchangeable random partition
distribution such that:
\begin{equation}
\label{eq_nonuniform}
s_{ik} \le s_{i'k'} \quad \implies \quad \text{pr}(c_i = c_k) \ \le \ \text{pr}(c_{i'} = c_{j'}).
\end{equation}
That is, the probability that items $i$ and $k$ cluster should be less
than that of items $i'$ and $k'$ if the proximity of $i$ and $k$ is less than
that of $i'$ and $k'$.

\subsection{Attractions from Proximities and a Permutation}
\label{sec_proposal_proximities}

Our proposed random partition distribution is given in terms of attractions
defined from the proximity information and a given permutation $\bm{\sigma}$.
Later we discuss how we choose the permutation or consider all possible
permutations.  For now, it suffices to say that attractions are permutation
specific.  Let $h_{ik}$ denote the attraction of item $i$ to item $k$, defined
as:
\begin{equation}
\label{eq_attraction}
h_{ik} = 
\begin{cases}
\ ( \sigma^{-1}_i - 1 ) \ s_{ik}^t \ \big/ \, s_{i\cdot}^{\hspace{0.15ex}t} \quad & \text{if} \ \sigma^{-1}_k < \sigma^{-1}_i \\
\ 0 & \text{otherwise}
\end{cases}
\end{equation}
where $s_{i\cdot}^t$ is the sum of proximities (that have been raised to the power of $t>0$) between $i$
and all items in the permutation before $i$, i.e.:
\begin{equation}
s_{i\cdot}^{\hspace{0.15ex}t} \ = \sum_{k \ : \ \sigma^{-1}_k \ < \ \sigma^{-1}_i} s_{ik}^t,
\end{equation}
where, as defined previously, $\sigma^{-1}_k$ is the order in which item $k$ is
assigned a cluster.  Thus, the sum of all attractions of $i$ to all the
preceeding items is $\sigma^{-1}_i - 1$, i.e., the number of preceeding items.
If $k$ preceeds $i$ in the permutation $\bm{\sigma}$, then the attraction of
$i$ to $k$ is proportional to the proximity $s_{ik}$ raised to the power of
$t$.  Note that the attractions are not symmetric (i.e., $h_{ik} \not= h_{ki}$)
and, if $i$ preceeds $k$ in the permutation $\bm{\sigma}$, there is no
attraction of $i$ to $k$ (i.e., $h_{ik} = 0$).  We call $t>0$ in the exponent
the \emph{temperature}, as it has the effect of dampening or accentuating the
proximity information.

\subsection{Chinese Restaurant Attraction Distribution for Given Permutation and Temperature}
\label{sec_cra_given}

We now present our proposed Chinese restaurant attraction distribution for the
given permutation $\bm{\sigma}$ used to define the attractions from the known
proximities and given temperature.  Its probability mass function for a cluster
label vector $\bm{c}$, given the permutation $\bm{\sigma}$ and temperature $t$,
is is the product of increasing conditionals:
\begin{equation}
\label{eq_cra_ppf}
p(\bm{c} \given \bm{\sigma}, t) = \prod_{i=1}^n p(c_{\sigma_i} \given c_{\sigma_1}, \ldots, c_{\sigma_{i-1}}, t)
\end{equation}
where:
\begin{equation}
\label{eq_cra_ppf_component}
\begin{split}
p(c_{\sigma_i} \given c_{\sigma_1},\ldots,c_{\sigma_{i-1}}, t) 
  & \ = \ \text{pr}(c_{\sigma_i} = j \given c_{\sigma_1},\ldots,c_{\sigma_{i-1}}, t) \\
  & \ \propto \
\begin{cases}
\ \sum_{k=1}^{t-1} \ h_{ik} \cdot \text{I}\{c_{\sigma_k} = j\} & \text{for\
\ } j=1,\ldots,q_{i-1} \\
\ \ \mass & \text{for\ \ } j = q_{i-1} + 1.
\end{cases}
\end{split}
\end{equation}
Note that (\ref{eq_dpm_ppf}) and (\ref{eq_cra_ppf}) are the same except
(\ref{eq_cra_ppf}) conditions on both a permutation $\bm{\sigma}$ and a
temperature $t$ (through the definition of the attraction $h_{ik}$ in
(\ref{eq_attraction})).  The difference between (\ref{eq_dpm_ppf_component})
and (\ref{eq_cra_ppf_component}) is that (\ref{eq_cra_ppf_component})
introduces an attraction $h_{ik}$ (which implicitly depends on the temperature
$t$) in front of the indicator function.  Note that the normalizing constant is
still $1/( \mass + i - 1 )$.  Thus, the probability of assigning item
$\sigma_i$ to a new cluster is still $\mass / ( \mass + i - 1)$.

Because of the equivalence of a cluster label vector $\bm{c}$ and a partition $\partition_n$,
we can use (\ref{eq_cra_ppf}) and (\ref{eq_cra_ppf_component}) to express
$p(\partition_n \mid \bm{\sigma}, t)$,
the probability mass function of a partition $\partition_n$ for the Chinese restaurant attraction distribution:
\begin{equation}
\label{eq_cra}
p(\partition_n \mid \bm{\sigma}, t) = \prod_{S \in \partition_n} \, \mass \, \biggg[ \prod_{i \in S} \sum_{k=1}^n h_{ik} \cdot \text{I}\{ k \in S \} \biggg] \Biggg/ \mass^{(n)},
\end{equation}
where we interpret the expression in the square brackets to be 1 if the size of
cluster $S$ is 1.  This is the same expression as (\ref{eq_dpm}), except the
gamma function is replaced with the more complicated expression in square
brackets and the function depends on the permutation $\bm{\sigma}$ and
temperature $t$ (through the definition of the attraction $h_{ik}$ in
(\ref{eq_attraction})).

\subsection{Marginalized Chinese Restaurant Attraction Distributions through Priors and Integration}
\label{sec_mcra}

So far, the Chinese restaurant attraction distribution has been defined
conditioned on a permutation $\bm{\sigma}$ and temperature $t$.  The
distribution of the permutation $\bm{\sigma}$ is a modeling choice.  Among the
possibilities, we discuss three.  First, if there is a natural sequential
ordering of the observations, the permutation could be fixed at that value,
i.e., the distribution of the permutation $\bm{\sigma}$, denoted
$p(\bm{\sigma})$, could simply be a point mass distribution at the one natural
permutation.  Second, one could use a point mass distribution at the
permutation $\bm{\sigma}$ that allocates items from largest average proximity
to the smallest average proximity, thereby assigning early customers who are
close on average and assigning later those that are more extreme.  The rational
here is to minimize the influence of outliers in initial cluster formation.
This ordering is likely to be unique if there are no ties among the
proximities.  Finally, one might consider using the uniform distribution on
permutations, i.e., $p(\bm{\sigma}) = 1/n!$ for all $\bm{\sigma}$.  Since the
temperature $t$ is restricted to the positive real line, an obvious
distribution for $t$, with density $p(t)$, is a gamma distribution.

With the known proximity or distance information, we integrate over the random
permutation $\bm{\sigma}$ and random temperature $\bm{t}$, to obtain the the
marginalized Chinese restaurant attraction (MCRA) distribution, whose
probability mass function is:
\begin{equation}
\label{eq_mcra}
p(\partition_n) = \int \int \ p(\partition_n \given \bm{\sigma}, t ) \ p(\bm{\sigma}) \ p(t) \ d \bm{\sigma} \ d t.
\end{equation}
Equivalently, the MCRA distribution $p(\partition_n)$ for $\partition_n$ can be expressed as: 
\begin{equation}
\label{eq_bnp_dpm}
\begin{split}
\partition_n \given \bm{\sigma}, t \ & \sim p(\partition_n \given \bm{\sigma}, t) \\
\bm{\sigma} & \sim p(\bm{\sigma}) \\
t & \sim p(t),
\end{split}
\end{equation}
where $p(\partition_n \mid \bm{\sigma}, t)$ is (\ref{eq_cra}), $p(\bm{\sigma})$
is a distribution on the integers $1,\ldots,n$ (e.g., a uniform distributions),
and $p(t)$ a distribution on the positive real line (e.g., a gamma
distribution).  In Section \ref{sec_modeling}, we propose the MCRA distribution
as a prior distribution on a partition $\partition_n$ in Bayesian modeling.
Before doing so, we explore the properties of the MCRA distribution in Section
\ref{sec_properties}.

\section{Properties of Proposed Distribution}
\label{sec_properties}

\subsection{Distribution of the Number of Clusters}
\label{sec_dist_number_of_clusters}

In both the Chinese restaurant process (see (\ref{eq_dpm_ppf}) and
(\ref{eq_dpm_ppf_component})) and the proposed Chinese restaurant attraction
distribution (see (\ref{eq_cra_ppf}) and (\ref{eq_cra_ppf_component})), the
probability that item $\sigma_i$ is assigned to a new cluster is $\mass / (
\mass + i - 1)$.  Therefore, the distribution of the number of clusters in
(\ref{eq_dist_number_of_clusters}) and the the mean and the variance in
(\ref{eq_moments_number_of_clusters}) hold in both cases.  The point is that
the mass parameter $\mass$ and pairwise proximity information are orthogonal.
The role, interpretation, and intuition regarding the mass parameter $\mass$
that one has for the Chinese restaurant process carries over directly to the
proposed Chinese restaurant attraction distribution.  Further, the pairwise
proximity information influences the allocation of probability among partitions
within a \emph{given number} of components, but does \emph{not} shift
probability among sets of partitions of \emph{different} number of components.

\subsection{Exchangeability}

A feature of the proposed random partition distribution is that it is
\emph{not} exchangeable.  For a fixed number of components and size of each
component, partitions which concur with the known pairwise proximity
information are given more probability than are partitions which conflict with
the pairwise proximity information.

While nonexchangeable, it should be noted that our proposal is invariant to the
order in which the observations are recorded in the dataset.  Consider a
cluster label vector $\bm{c}$, a permutation $\bm{\sigma}$, and known pairwise
proximities.  An arbitrary permutation $\bm{c}^\ast$ of the elements of
$\bm{c}$ has the same probability as does $\bm{c}$, assuming that same
permutation is also applied to the permutation $\bm{\sigma}$ and the known
proximities.

In contrast to order invariance, our proposal is permutation dependent in that
the probability of a partition $\partition_n$ in (\ref{eq_cra}) depends on the
permutation $\bm{\sigma}$.  Thus, except for special cases discussed in Section
\ref{sec_special_cases}, for any two permutations $\bm{\sigma}^{(1)} \not=
\bm{\sigma}^{(2)}$ among all possible $n!$ permutations and any partition
$\partition_n$, it may well be that:
\begin{equation}
\label{eq_permutation_invariance}
p(\partition_n \given \bm{\sigma}^{(1)}) \not= p(\partition_n \given \bm{\sigma}^{(2)}).
\end{equation}
Hence choosing different distributions for $\bm{\sigma}$ leds to different
marginalized Chinese restaurant attraction (MCRA) distributions in
(\ref{eq_mcra}).

\subsection{Consistency}

In constrast with the Chinese restaurant process, the proposed Chinese
restaurant attraction process is \emph{not} consistent.  Consistency requires
that the marginal probabilities that two items are clustered together is
constant regardless of what other items (with their associated pairwise
proximity information) are considered.  Of course, saying that two items are
``distant'' is relative to other items considered and, hence, subject to change
as more items are considered.  Therefore, consistency should not be expected in
the presence of pairwise proximity information.

More formally, we argue that insisting on consistency in the context of
pairwise proximity information is too limiting.  We do so with a simple example
of $n=3$ items.  Consider the distribution $p_0$ of the partition
$\partition_3$ in (\ref{eq_cra}).  Let $p_1$ be the distribution of the
partition $\partition_2$ obtained by marginalizing $p_0$ over item $3$:
\begin{align*}
p_1(\{\{1,2\}\}) & = p_0(\{\{1,2,3\}\}) + p_0(\{\{1,2\},\{3\}\}) \\
p_1(\{\{1\},\{2\}\}) & = p_0(\{\{1,3\},\{2\}\}) + p_0(\{\{1\},\{2,3\}\}) +
p_0(\{\{1\},\{2\},\{3\}\})
\end{align*}
Finally, let $p_2$ be the distribution
of the partition $\partition_2$ in (\ref{eq_cra}) assuming $n=2$.  Consistency
requires that $p_1$ and $p_2$ be the same distribution:
\begin{equation}
\label{eq_consistency}
\begin{split} p_1(\{\{1,2\}\}) & = p_2(\{\{1,2\}\}) \\
p_1(\{\{1\},\{2\}\}) & = p_2(\{\{1\},\{2\}\}).
\end{split}
\end{equation}
That is, consistency requires that:
\begin{equation}
\label{eq_consistencyfull}
\begin{split}
p_0(\{\{1,2,3\}\}) + p_0(\{\{1,2\},\{3\}\}) & = p_2(\{\{1,2\}\})
\\ p_0(\{\{1,3\},\{2\}\}) + p_0(\{\{1\},\{2,3\}\}) + p_0(\{\{1\},\{2\},\{3\}\})
& = p_2(\{\{1\},\{2\}\}).
\end{split}
\end{equation}

We now show that requiring consistency --- in particular, that
(\ref{eq_consistencyfull}) holds --- leads to a severe constraint on the
possible proximities between items.  Without loss of generality, assume that
the temperature $t$ is fixed at $1$, the mass parameter $\mass$ is fixed at
$1$, and that proximity between items $1$ and $2$ is $1$. Let the proximity
between items $1$ and $3$ be $a$ and the proximity between items $2$ and $3$ be
$b$.  (Hence the distances between items $1$ and $2$ and between items $2$ and
$3$ are, respectively, $1/a$ and $1/b$.) Consider the marginalized Chinese
restaurant attraction distribution with a uniform distribution on the six
possible permutations of the $n=3$ items.  Given the small sample size, it is
possible to symbolically find the probabilities for each partition.  Table
\ref{tab_distributions} shows the probabilities for partitions from
distributions $p_0$ and $p_2$, as well as the relationship in
(\ref{eq_consistencyfull}) required by consistency.  It can be shown that in
order for the probabalities in Table \ref{tab_distributions} to satisfy
(\ref{eq_consistencyfull}), the proximities $a$ and $b$ must be such that:
\begin{equation}
\label{eq_constraint}
b = \frac{1}{a}.
\end{equation}
This constraint is displayed graphically in Figure \ref{fig_cassini}, where the
Cassini oval shows the distances among the three observations that give
consistency.  The conclusion is that requiring consistency severely constrains
the pairwise information in ways that not are likely to be seen in practice.
Incidently, the probabilities in Table \ref{tab_distributions} can be used to
verify the desired property in (\ref{eq_nonuniform}).

\setlength{\unitlength}{.001\textwidth}%{.025in}
\begin{table}[tb]
\caption{Tables of probabilities for each partition for distributions $p_0$ and
$p_2$ and an illustration that consistency requires that the probabilities of
the first two partitions from $p_0$ must be equal to $1/2$ and the sum of the
probabilities of other three partitions from $p_0$ must also be $1/2$}
\label{tab_distributions}
\centering
\newcommand{\mybox}[1]{\parbox[t][5ex][c]{16ex}{#1}}
\newcommand{\myboxx}[1]{\parbox[t][5ex][c]{25ex}{#1}}
\newcommand{\myboxxx}[1]{\parbox[t][5ex][c]{5ex}{#1}}
\newcommand{\myboxxxx}[1]{\parbox[t][12ex][c]{5ex}{#1}}
\begin{picture}(850,400)
%\put(0,0){\line(1,0){850}}
%\put(0,0){\line(0,1){400}}
%\put(850,0){\line(0,1){400}}
%\put(0,400){\line(1,0){850}}
\put(0,0){\makebox(400,400)[tl]{
  \small
  \begin{tabular}{ll}
  \toprule
  Partition $\partition_3$ & Probability from $p_0$ \\ \midrule
  \mybox{$\{\{1,2,3\}\}$} & \myboxx{$ \dfrac{1}{3} $} \\
  \mybox{$\{\{1,2\},\{3\}\}$} & \myboxx{$ \dfrac{1}{18} + \dfrac{1}{9}\bigg( \dfrac{1}{1+a} + \dfrac{1}{1+b} \bigg) $} \\
  \mybox{$\{\{1,3\},\{2\}\}$} & \myboxx{$ \dfrac{1}{18} + \dfrac{1}{9}\bigg( \dfrac{a}{a+b} + \dfrac{a}{1+a} \bigg) $} \\
  \mybox{$\{\{1\},\{2,3\}\}$} & \myboxx{$ \dfrac{1}{18} + \dfrac{1}{9}\bigg( \dfrac{b}{a+b} + \dfrac{b}{1+b} \bigg) $} \\
  \mybox{$\{\{1\},\{2\},\{3\}\}$} & \myboxx{$ \dfrac{1}{6} $} \vspace{1ex} \\
  \bottomrule
  \end{tabular}
}}
\put(505,68){\resizebox{3ex}{11.2ex}{\}}}
\put(505,247){\resizebox{3ex}{8ex}{\}}}
\put(560,0){\makebox(200,400)[tl]{
  \small
  \begin{tabular}{ll}
  \toprule
  Partition $\partition_2$ \hspace{2ex} & Probability from $p_2$ \\ \midrule
  \myboxxxx{$\{\{1,2\}\}$} & \myboxxxx{$\dfrac{1}{2}$} \\
  \myboxxx{} & \myboxxx{} \\
  \myboxxx{$\{\{1\},\{2\}\}$} & \myboxxx{$\dfrac{1}{2}$} \\
  \myboxxx{} & \myboxxx{} \vspace{1ex} \\
  \bottomrule
  \end{tabular}
}}
\end{picture}
\end{table}

\begin{figure}[b]
\centering\includegraphics[width=0.3\textwidth]{cassini-oval}
\caption{In order for the proposed distributions to have the consistency
property, item 3 is constrained to fall on this Cassini oval.}
\label{fig_cassini}
\end{figure}

\subsection{Simplification in Special Cases}
\label{sec_special_cases}

We note that the proposed marginalized Chinese restaurant attraction
distribution simplifies to the usual Chinese restaurant process in two cases.
The first case is when all items are equi-distant to all other items, e.g.,
$s_{ik}$ equals a constant for all pairs $i$ and $k$.  The second case is when
the temperature $t$ is equal to 0.  In this case $s_{ik}^t$ equals $1$,
regardless of the value $s_{ik}$, and therefore $h_{ik} = 1$, making
(\ref{eq_cra_ppf_component}) reduce to (\ref{eq_dpm_ppf_component}).

\subsection{Effect of Temperature}
\label{sec_temperature}

To illustrate the effect of temperature $t$ in the marginalized Chinese
restaurant attraction distribution, consider the ``USArrests'' dataset from R
\citep{Rsoftware}.  The USArrests datasets ``contains statistics, in arrests
per 100,000 residents for assault, murder, and rape in each of the 50 United
States.  in 1973.  Also given is the percent of the population living in urban
areas.'' We computed the Euclidean distance between the four-dimensional data
vectors of five randomly selected states.  The data and pairwise distances are
displayed in Table \ref{tab:usa}.

\begin{table}
\centering
\caption[Subset of the USArrests dataset]{
Five-state subset of the USArrests dataset and their associated Euclidean distances}
\label{tab:usa}
\begin{subtable}[b]{0.48\textwidth}
\footnotesize
\begin{tabular}{lrrrr}
  \toprule
\multicolumn{2}{r}{Murder} & Assault & Rape & Urban \\
  \midrule
  North Carolina (NC) & 13.00 & 337 &  45 & 16.10 \\ 
  Illinois (IL) & 10.40 & 249 &  83 & 24.00 \\ 
  Nebraska (NE) & 4.30 & 102 &  62 & 16.50 \\ 
  Michigan (MI) & 12.10 & 255 &  74 & 35.10 \\ 
  Washington (WA) & 4.00 & 145 &  73 & 26.20 \\ 
  \bottomrule
\end{tabular}
\caption{Subset of USArrests dataset}
\label{tab:usa:data}
\end{subtable}
\hspace{2ex}
\begin{subtable}[b]{0.48\textwidth}
\footnotesize
\begin{tabular}{rrrrrrr}
  \toprule
  & NC & IL & NE & MI & WA \\ 
  \midrule
  NC & 0.00 & 96.21 & 235.77 & 89.03 & 194.50 \\ 
  IL & 96.21 & 0.00 & 148.81 & 15.59 & 104.70 \\ 
  NE & 235.77 & 148.81 & 0.00 & 154.79 & 45.43 \\ 
  MI & 89.03 & 15.59 & 154.79 & 0.00 & 110.66 \\ 
  WA & 194.50 & 104.70 & 45.43 & 110.66 & 0.00 \\ 
  \bottomrule
\end{tabular}
\caption{Euclidean distances between data vectors}
\label{tab:usa:dist}
\end{subtable}
\end{table}

We computed the probability of each of the $B(5) = 52$ possible parititions of
the five states for a range of temperatures, letting the mass parameter
$\mass=2$ and using the uniform distribution on permutations.  The results are
displayed in Figure \ref{fig_evolution_joint}.  At a given temperature, the
cumulative probabilities of the $52$ partitions for the five states are
displayed.  For each partition, the cumulative probabilities across
temperatures are joined to form the curves and the probability of a given
partition is the difference between curves.  The curves of several interesting
partitions are identified using the cluster label notation.  The temperature
$t=0$ corresponds to the partition distribution of the usual Chinese restaurant
process.  As the temperature increases, the pairwise distances become more
influential.  That is, as the temperature gets large, the marginalized Chinese
restaurant attraction distribution has appreciable probability on several
partitions and virtually no probability on others.  Partitions that lose the
most probability are shaded pink and those that gain the probability are
labeled and shaded cyan.  Note, however, that the total probability of
partitions with a given number of clusters stays constant.  (See Section
\ref{sec_dist_number_of_clusters}.)

\begin{figure}[tb]
\centering\includegraphics[width=0.95\textwidth]{evolution-joint}
\caption{The cumulative probabilities of the $52$ partitions for the five
states selected from the USArrests dataset.  For each partition, the cumulative
probabilities across temperatures are joined to form the curves and the
probability of a given partition is the difference between curves.  The curves
of several interesting partitions are identified using the cluster label
notation.}
\label{fig_evolution_joint}
\end{figure}

Whereas, Figure \ref{fig_evolution_joint} shows the partition distribution
evolving across temperature $t$, Figure \ref{fig_evolution_pairwise} shows the
evolution across temperature $t$ of the marginal probability of two states
clustering.  When temperature $t=0$, the marginal probability of any two states
clustering is $1/(1+\mass) = 1/3$ and these probabilities diverge --- and
eventually asymptote --- as the temperature $t$ increases.

\begin{figure}[tb]
\centering\includegraphics[width=0.65\textwidth]{evolution-pairwise}
\caption{Evolution across temperature $t$ of the marginal probability of two states
clustering in the USArrests example.}
\label{fig_evolution_pairwise}
\end{figure}

\section{As a Prior Distribution on a Partition in Bayesian Modeling}
\label{sec_modeling}

\subsection{Model}

Using a Dirichlet process \citep{ferg:1973} as a prior distribution for an
unknown mixing distribution in mixture models was first suggested by
\citet{anto:1974}.  Its use leads to arguably the most popular Bayesian
nonparametric model: the Dirichlet process mixture model.  The Dirichlet
process implies the Polya urn scheme \citep{blac:macq:1973}, a probability
model $p(\partition_n)$ for a random partition $\partition_n$ equivalent to Ewens-Pitman
distribution \citep{ewens:1972,pitm:1995,pitm:1996}.
A broader class of Bayesian nonparametric models for data $\bm{y} =
(y_1,\ldots,y_n)$ can be written as:
\begin{equation}
\label{eq_bnp_dpm}
\begin{split}
y_i \given \theta_i \ & \sim \ p(y_i \given \theta_i) \\
\theta_i \ & = \ \sum_{j=1}^{q_n} \phi_j \I{i \in S_j} \\
\phi_j \ & \sim \ G_0 \\
\partition_n \ & \sim \ p(\partition_n),
\end{split}
\end{equation}
where $p(y \given \theta)$ is a sampling distribution indexed by $\theta$,
$G_0$ is the prior distribution from which $\phi_1,\ldots,\phi_{q_n}$ are
independently drawn, and $p(\partition_n)$ is a prior distribution over
partitions of the form $\partition_n = \{S_1,\ldots,S_{q_n}\}$.  Let $\bm{\phi}
= (\phi_{q_n},\ldots,\phi_{q_n})$.  The model can be enriched by indexing the
sampling model by other parameters or by placing hyperpriors on quantities
defining the prior distribution.  When $p(\partition_n)$ is defined as in
(\ref{eq_dpm}), $G_0$ is known as the ``centering'' distribution of the
Dirichlet process and the model is the popular Dirichlet process mixture model.
When pairwise information is present, however, we propose that the prior
distribution $p(\partition_n)$ be the marginalized Chinese restaurant
attraction distribution, given in (\ref{eq_mcra}).

\subsection{Sampling from the Prior Distribution of a Random Partition}

We now describe how to sample a partition $\partition_n$ from the marginalized
Chinese restaurant attraction distribution.  First, if the permutation is not
fixed, sample a permutation $\bm{\sigma}$ from the permutation distribution
$p(\bm{\sigma})$ discussed in Section \ref{sec_mcra}.  In the case of the
uniform distribution on partitions, i.e., $p(\bm{\sigma}) = 1/n$, a random
permutation is readily obtained by sorting integers $1,\ldots,n$ according to
uniformally-distributed random numbers on the real line or through standard
functions in software (e.g., \texttt{sample(1:n)} in R).  Next, if it is not
fixed, sample a temperature $t$ from the distribution $p(t)$, e.g., a gamma
distribution.  Given the permutation $\bm{\sigma}$ and temperature $t$,
calculate the attraction matrix as described in Section
\ref{sec_proposal_proximities}.  Finally, sample the partition $\partition_n$
from (\ref{eq_cra_ppf}) using the increasing conditionals in
(\ref{eq_cra_ppf_component}).  Multiple partitions from the marginalized
Chinese restaurant attraction distribution can be obtained by repeating this
sampling process many times, computing new attractions at each iteration.

\subsection{Sampling from the Posterior Distribution of a Random Partition}

In Bayesian analysis, interest lies in the posterior distribution of parameters
given the data.  The posterior distribution $p(\partition_n, \bm{\sigma}, t,
\bm{\phi} \given \bm{y})$ is not available in closed-form, but a Markov chain
Monte Carlo (MCMC) algorithm is available, as we now describe.  This algorithm
systematically updates parts of the parameter space at each iteration and
performs many iterations.

First, we present the update of $\bm{\phi} = (\phi_{q_n},\ldots,\phi_{q_n})$
given $\partition_n$, $\bm{\sigma}$, $t$.  This update is the same as in any other random partition
model, including the Dirichlet process mixture model.  Let $\bm{y}_j = \{y_i :
i \in S_j\}$ for the partition $\partition_n = \{ S_1, \ldots, S_{q_n} \}$.
For $j=1,\dots,q_n$, update $\phi_j$ using its full conditional distribution:
\begin{equation}
p(\phi_j \given \bm{y}_j) \propto \ p(\phi_j) \prod_{i \in S_j} p(y_i | \phi_j),
\end{equation}
where $p(\phi)$ is the density of the centering distribution $G_0$.  This full
conditional distribution can usually be sampled
directly if $G_0$ is conjugate to the sampling model $p(y \given \phi)$.  If not,
any other valid MCMC update can be used, including a Metropolis Hastings update.

Second, consider the update of the partition $\partition_n$ given
$\bm{\sigma}$, $t$, and $\bm{\phi}$.  Because the model is not exchangeable,
the usual algorithms (such as those reviewed in \citet{neal:2000}), do not
hold.  Let $\mathcal{C}_i(\partition_n)$ be the set of partitions that contains
$\partition_n = \{ S_1, \ldots, S_{q_n}\}$ and all partitions that can be obtained from $\partition_n$
by reallocating item $i$ to a new singleton component $S_{q_n+1}$ or one of the
existing components $S_1,\ldots,S_{q_n}$.

\[
p( \ i \in S_j \given \partition_n, t, \bm{\sigma}, \bm{\phi} ) \propto \prod_{n=1}^q
\]

\[
\{ S_1 \setminus \{i\}, \ldots, S_{q_n} \setminus \{i\} \}
\]

%    \subsection{Sampling from the Posterior Distribution}
%    \DBD{---Posterior sampling using MCMC.
%    Three challenges: 1. Updating the partition (show how to modify Neal's Auxliary Gibbs sampler (Algorithm 8)),
%                      2. Updating the Permutation,
%                      3. Updating the temperature. Give proposal mechanism, etc.}
%    
%    \section{Demonstration}
%    
%    \subsection{``Knob-Socket'' Model for Protein Structure Prediction}
%    
%    \subsection{Computational Considerations}
%    
%    \DBD{---CPU time and acceptance rates.  How it scales.}
%    
%    \subsection{Model Comparison Results}
%    
%    \section{Discussion}
%    
%    \DBD{``This is the concluding part of the paper. It is only needed if it contains new material. It should
%    not repeat the summary or reiterate the contents of the paper.''}

\section*{Acknowledgments}

The authors gratefully acknowledge Peter M\"{u}ller, David H.\ Russell, Lei Tao, Ryan Day, Gordon B.\ Dahl,
Kristin P.\ Lennox, and Marina Vannucci for helpful suggestions.
This work has been supported by NIH NIGMS R01 GM104972.

\bibliography{refs}
\bibliographystyle{biometrika}

% \input{old}

\end{document}

\documentclass{article}
\usepackage{fullpage}
\usepackage{mathtools}
\usepackage{bm}
\linespread{.95}
\def\wl{\par \vspace{\baselineskip}}

\begin{document}

\begin{center}
  \section*{Variable Selection for Random Partition Models}
  {\normalsize \textbf{Arthur Lui~~~~~~~luiarthur@gmail.com~~~~~~~arthurll}}\\
  {\footnotesize Mentor: David B. Dahl, Department of Statistics, Brigham Young University}
\end{center}

\section*{\normalsize Project Purpose}
The purpose of this project is two-fold: 
(1) Develop a novel statistical methodology for Bayesian nonparametrics,
(2) Apply this methodology to predict secondary structure in proteins
given their primary sequences.

\section*{\normalsize Project Importance}
  As noted by Mardia (2013), ``Proteins are the workhorses of all living systems, 
  and protein bioinformatics deals with analysis of protein 
  sequences (one dimensional) and structures (three dimensional).''
  The three-dimensional shape of a protein determines its functionality.
  Thus, understanding the relationship between the 
  primary and secondary structure of a protein is a fundamental importance
  in the biological sciences, facilitating the process
  of drug discovery and the understanding of evolutionary biology.
  The new statistical methodology that we will develop will help
  model the relationship between primary and secondary structure of a protein.
  In particular, we will explore the statistical merit of proposed biochemistry
  variables in modeling secondary structure based on random partition
  models.

\section*{\normalsize Project Profile Body}
  This section outlines the statistical approach we will pursue to accomplish
  the two-fold project purpose outlined above.  A fundamental idea in our
  approach is exploiting a latent clustering structure to borrow information
  in parameter estimation.  A clustering structure can be encoded using
  a partition $\pi_n = \{S_1,...,S_{q_n}\}$ of $\{1,...,n\}$.  A partition has the
  following properties: (i) $S_i \neq \emptyset$ for $i=1,...,q_n$ (i.e.,
  non-empty subsets), (ii) $S_i \cap S_j = \emptyset$ for $i \neq j$ (i.e.,
  mutually exclusive subsets), and (iii) $\cup_{j=1}^q S_j = \{1,...,n\}$
  (i.e., exhaustive subsets).  The number of subsets $q_n$ for a partition
  $\pi_n$ can range from 1 (i.e., all items belong to the same subset) to
  $n$ (i.e., each item is in a singleton subset).
  It is sometimes convenient to represent a partition $\pi_n$ as a vector 
  $\bm{c} = (c_1,...,c_n)$ of cluster labels, where $c_i = j$ if and only if 
  $i \in S_j$, for $i=1,...,n$ and $j=1,...,q_n$.  The size of $\mathcal{F}_n$ 
  grows exponentially in $n$ according to the Bell (1934) number.
  A probability distribution over all partitions is discrete, 
  but the size of the space makes exhaustive calculations impossible 
  except for very small $n$.

  The Chinese restaurant process (CRP) is the prototypical random partition model. 
  The exchangeability of the CRP establishes that the
  marginal probability that any two items $i$ and $k$ are clustered together is:
  \begin{equation*}
  \label{eq_uniform}
  \text{pr}(c_i = c_k ) = \frac{1}{\alpha+1},
  \end{equation*}
  where $\alpha$ is a mass parameter governing the degree of clustering.
  Note that this probability is uniform across all pairs $i$ and $k$.  The probability
  mass function of a partition $\pi_n$ for the Chinese restaurant process
  is:
  \begin{equation*}
  \label{eq_dpm}
  p(\pi_n) = \prod_{S \in \pi_n} \alpha \Gamma(|S|) / \alpha^{(n)},
  \end{equation*}
  where $\alpha^{(n)} = \prod_{i=1}^n(\alpha+i-1)$ is the reciprocal of the
  normalizing constant, $\Gamma(x)$ is the gamma function, and $|S|$ is
  the number of items in subset $S$.

  Dahl (2013) proposed a random partition distribution indexed by pairwise information.
  He supposed that the prior information regarding clustering of $n$ items can be
  expressed in terms of an $n \times n$ proximity matrix of elements $s_{ik}$
  giving the proximity between items $i$ and $k$.  If pairwise information
  regarding $i$ and $k$ is more readily expressed in terms of a distance
  $d_{ik}$, then the proximity $s_{ik}$ can be expressed as a function of
  $d_{ik}$, e.g., $s_{ik} = 1/d_{ik}$ or $s_{ik} = \max_{\forall ab}\{d_{ab}\} -
  d_{ik}$.  The framework can use any proximity or distance metric such that $0 <
  s_{ik} < \infty$ for all $i,k$.  Dahl (2013) defined a nonexchangeable random partition
  distribution such that:
  \begin{equation}
  \label{eq_nonuniform}
  s_{ik} \le s_{i'k'} \quad \implies \quad \text{pr}(c_i = c_k) \ \le \ \text{pr}(c_{i'} = c_{j'}).
  \end{equation}
  That is, the probability that items $i$ and $k$ cluster should be less
  than that of items $i'$ and $k'$ if the proximity of $i$ and $k$ is less than
  that of $i'$ and $k'$.

  Rather than using a single covariate to define the proximities for parameter
  estimation, the purpose of this ORCA-funded research is to investigate a random
  partition model indexed by pairwise information that incorporates multiple
  covariates.  The clustering underlying the parameter estimation can then be
  influenced by several covariates.  We propose to investigate methodologies to
  perform variable selection to find the important covariates and weight them for
  optimal parameter estimation.

  Consider again the protein secondary structure prediction problem, where amino
  acids that have similar hydrophobicity may share similar parameter values.  We
  hypothesize that having the clustering influenced by hydrophobicity will improve
  the parameter estimation.  Likewise, polarity is another covariate that may be
  related to the parameter estimation.  In all, we hypothesize that there are many
  covariates (together with their interactions) that may be related to the
  parameter estimation.  In our proposed method, we wish to cluster amino acids
  according to multiple proximity measures (hydrophobicity, polarity, charge,
  etc.), automatically selecting and weighting those that help in parameter
  estimation.

%  Using regression, we will obtain a weights for each distance matrix 
%  for multiple covariates. The distance matrices for each of the covariates 
%  will be the regressors in the regression. Using the weighted sum of the distance 
%  matrices, we can then cluster proteins that have higher proximity, 
%  or smaller distance, according to the new distance matrix.

\section*{\normalsize Anticipated Academic Outcome}
  The ultimate goal is to publish an article in a statistics journal. In the process of
  doing so, I will present my research at the BYU's College of Physical and Mathematical Science
  Spring Research Conference in 2014. I will also present my research 
  at the Joint Statistical Meeting (JSM) in Boston in August 2014.

\section*{\normalsize Qualifications}
  My mentor, Dr. Dahl, is collaborating with other statisticians 
  and biochemists on a long-term project developing Bayesian nonparametric 
  methods for protein structure prediction.
  I will be able to draw help from Dr. Dahl 
  as his research emphasis is also on Bayesian nonparametrics.

  I am a senior majoring in statistics. In addition to taking required 
  statistics and math courses, I have also taken computer 
  science and extra math courses to further develop my skills. 
  I have also taken a molecular biology course and several chemistry 
  courses, including organic chemistry. As such, I am able to read 
  relevant material about biochemistry and gain sufficient 
  understanding to conduct the proposed research.

\section*{\normalsize Project Timetable}
  \begin{tabular}{ll}
    December 2013&Develop the initial methodology\\
    January 2014&Implement the algorithm in Scala\\
    March 2014&Present at CMPS Spring Research Conference\\
    August 2-7, 2014&Present at JSM 2014 (Boston)\\
    August 31, 2014&Submit an article in a statistics journal
  \end{tabular}

\section*{\normalsize Scholarly Sources}
  \textbf{Schmidler, S. C., Liu, J. S., and Brutlag, D. L.} (2000), ``Bayesian segmentation of protein secondary structure,'' Journal of Computational Biology, 7, 233-248.\\
  \textbf{Dahl, D. B., Li, Q. W., Vannucci, M., Joo, H., Tsai, J. W.} (2013), ``A Bayesian Model for Protein Secondary Structure Prediction,'' Proceedings of the ISI World Statistics Congress.\\
  \textbf{Dahl, D. B., Joo, H., Tsai, J. W.} (2013), ``Partition Distributions Indexed by Pairwise Information,'' \textit{working paper}.\\
  \textbf{Mardia, K. V.} (2013), ``Statistical approaches to three key challenges in protein structural bioinformatics''. Journal of the Royal Statistical Society: Series C (Applied Statistics), 62: 487–514.
\end{document}
